

https://boardgamegeek.com/article/17801232#17801232

Having played the Long Scenario 1-player many times, I always use the
historical step up (12 Cards from the 1964 deck, 24 Cards from the 1965 deck,
and 36 cards from the 1968). In constructing the Long Scenario deck, I start in
reverse order by first creating the stack that contains the 6th (Final) Coup
Card, and then placing on top of that stack the stacks containing the 5th
through 1st Coup Card. I start this process by shuffling the 6 Coup Cards, and
then laying them out from left to right. I then roll 2 die to pick the 6th
(Final) Coup Card. I then start with the Event Cards in the 1968 deck and deal
out 4 stacks of 12 Event cards (left to right) and then roll 2 die to determine
which of the four 1968 stacks is combined with the final Coup Card selected,
shuffle and then place that shuffled stack as the 6th (bottom) stack of the
Long Scenario deck. I repeat that process in selecting the 4th and 5th Coup
Cards (using 1 die only to pick the 5th Coup Card) and then to pick 2 of the 3
of remaining stacks of 1968 Event Cards. For the 1965 Event cards, I again deal
out four stacks of 12 Cards, but only select two of those stacks, one stack to
be combined with each of the 2nd and 3rd Coup Cards picked. Finally, I deal out
two stacks of 12 Event Cards each from the 1964, and pick one (again based on
rolling 2 die) to be combined with the 1st Coup Card which is shuffled and then
placed top of the other 5 stacks. By using this method, I feel I ensure as much
randomness in which Event Cards are chosen and in which order. Also, this
method for creating the deck doesn't take more than a couple of minutes to
execute.
