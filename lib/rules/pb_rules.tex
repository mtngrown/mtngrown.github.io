\documentclass{article}

\usepackage{hyperref}

\title{PanzerBlitz Rules of Play}

\begin{document}

\maketitle

\tableofcontents



\section{General Outline of Play}

Each side maneuvers its forces (playing pieces) on
the terrain map seeking to destroy the enemy units
and/or gain a specific territorial objective as outlined
in one of the scene-setting Situation Cards. Players
move their pieces and have combat by taking turns.
Each complete turn represents six minutes of real
time.

The forces in a given Situation may be unequal and
one side may have a better chance of winning than
the other, but it is primarily the skill of the individual
player which determines the outcome of the game.
The chance element introduced by the use of the
die-roll/Combat Results Table is only that degree of
chance consistently present in any real-life combat
event. The probabilities of combat outcomes have
been worked out using historical and technical data.

\section{The Mapboard}

The three section mapboard represents a varied
sampling of typical terrain in the Soviet Union. Each
section has a number (1, 2 or 3) located just above
the fold-line. The board is ``geo-morphic'' i.e. capable
of being changed by re-arranging the three sections
in a variety of juxtapositions. The long edge of each
two panel section will line up with the long edge of
any other section no matter which way they are
butted together. The short edges mate with each
other in a similar fashion.

\section{The Playing Pieces}

The square, cardboard pieces represent platoon or
company sized military units of several different
types (e.g. Infantry platoons, Tank platoons, Assault
Gun platoons, etc), which are the playing pieces
used in PANZERBLITZ. Hereafter they will be
referred to as ``units'' or ``unit counters''. The numbers
on the unit-counters represent that unit's capabilities
with respect to movement, attack, defense, and
range of weapons. The other symbols or silhouettes
identify what type of unit that counter represents.


Notice that all vehicle units are symbolized with an
appropriate silhouette and all other (non-vehicular)
units are symbolized with standard military-planning
symbols.

\section{Factor Definitions}

[COUNTER IMAGE HERE]

MOVEMENT FACTOR (MF) � The basic, maximum
number of hexagons (hexes) which a unit may move in
one turn. This capability can be reduced or increased
by terrain features.

ATTACK FACTOR (AF) � The basic offensive power of a
given unit.

DEFENSE FACTOR (DF) � The basic defensive power of
a given unit.

RANGE FACTOR (RF) � The maximum effective
distance (in hexagons) that a unit's Attack Factor can
be used against enemy units.

For example, a unit with a RF of 8 could fire its
weapons (use its attack factor) against any enemy unit
within that 8 hex range.

For a full display of all the counters in PanzerBlitz see
the Unit Identification Table.

Unit Identification Table and the Program
Identity Code System (PICS)

The Unit Identification Table shows a full ``breakout'' of
all the units in PanzerBlitz grouped according to
general category and function. Unless otherwise
stated, all Russian units are COMPANIES and all
German units are PLATOONS. The Unit Composition
portion of the table shows what actually went into the
make-up of the various units. (Note: Although the non-vehicular
units had organic transport assigned to
them, the game-counters symbolize them without this
transport � the trucks and wagons being given as
separate counters in each Situation.)

Each specific unit type has been assigned a specific
code number (example: Hetzer No. 832). Each
specific TYPE of unit has a common second digit
code number (example: all German ``Hetzers'' have
the number ``3'' as their second digit). Each
FUNCTIONAL CATEGORY is expressed by the left-
most number (example: all Tank Destroyers are in
the 800 series).

Roughly comparable Russian and German units
have been assigned the same TYPE code number
group. Towed guns have been assigned two-digit
numbers (although they may be thought of as having
a FUNCTIONAL CODE of ``0''). To fully identify a
given unit, write a ``G'' or ``R'' (nationally) followed by
its PIC number: Thus ``G/832'' means: German/TankDestroyer/
Hetzer/2nd Platoon.

Note: the right hand digit will only be a zero if there
is only one counter of that type supplied. The
second digit will only be a zero if there is only one
TYPE in that category.

The Program Identity Code system is simply a
shorthand method of positive unit identification for
use in play-by-mail games and noting the position of
units in games which must be interrupted and restarted.


\section{Movement}

\subsection{Terrain Features}

The hexagonal grid superimposed upon the
mapboard is used to determine movement and to
delineate the boundaries of the various terrain
features. A hex is considered to be a given type of
terrain if all or any part of it contains that terrain
feature. Terrain affects movement and defense as
outlined in the Terrain Effects Chart (TEC).

The ``heavy-hex-side'' symbols (different colored
bars superimposed upon some hex-sides) are
explained in the OBSTACLE AND ELEVATIONS
section of the rules.


The single hexagon on board section No. 3 which is
completely covered by the pond is unenterable for
ALL units. Units may use the hexes partially covered
by the pond.

The half-hexes on the outer edges of the board are
considered playable and may be utilized as if they
were complete hexagons.

\subsection{How to Move Units}

\begin{enumerate}

  \item[A -]
In any one turn a player may move as many or as
few of his units as he desires.


\item [B -]
Units which have fired (used their AF) may not
move in that turn. ``Dispersed'' units may not move
(see How to Have Combat and the Combat
Results Table).

\item [C -]
Units may move as much or as little as the player
desires within the limits of their MF's and the
terrain effects. Units with a MF of 1 may move one
hex per turn regardless of terrain.

\item [D -]
Units may move through friendly units.

\item [E -]
Units may not move through enemy units
(Exception: see OVERRUN RULE)

\item [F -]
Units may not stop on top of enemy units.

\item [G -]
There is no movement penalty when moving into
or through hexes adjacent to enemy units (i.e.
there is no ``zone of control'' such as in other
games.

\item [H -]
No enemy movement is allowed during friendly
movement.

\item [I -]
No combat, enemy or friendly, takes place during
movement (Exception: see OVERRUN RULE)

\end{enumerate}

\subsection{Transporting Units}

\begin{enumerate}
  \item [A -]
The ``C'' class vehicular units (wagons, trucks and
halftracks have the capability of carrying non-
vehicular units (guns, infantry and command
posts). Each ``C'' unit has the capacity to carry one
non-vehicular unit. To symbolize that a unit is a
passenger in a ``C'' unit, place the unit being
carried UNDER the ``C'' unit. Players should never
place non-vehicular units under vehicular units
unless they are being transported by that unit.

\item [B -]
In any one turn a ``C'' unit may either ``Load'' or
``Transport'' or ``Transport and Unload.'' It may only
perform one of these operations per turn.

\item [C -]
``C'' units and the unit(s) to be loaded must begin
their turn on the same hex. The passenger unit
may not fire (use its AF in the turn of loading, while
being transported or when unloading. Dispersed
units may not load or unload. Units are ``loaded''
when they are under the ``C'' unit.

\item [D -]
Passengers may not move independently on the
turn in which they unload from transporting units.

\item [E -]
A truck or wagon unit an the unit is transporting
have a combined defense factor of 1.

\item [F -]
When using halftracks as transport, the defense
factor (DF) of the halftrack unit is used when
attacked. Elimination affects both carrier and
passenger. The halftrack unit may fire while
loaded.

\item [G -]
Passengers and carriers are treated as one unit
for stacking purposes (see Stack Limitations).
Units are ``Loaded'' when under the ``C'' unit.
Passengers and their carriers are treated as one
unit for combat results purposes. If combat
results call for elimination, both are eliminated. If
dispersed, both are dispersed.

\item [H -]
Armored vehicle units (tanks, assault guns, etc.)
may carry non-vehicular units in a fashion
similar to ``C'' units. Each armored unit may carry
one passenger unit.

\item [I -]
If an armored unit is destroyed while
transporting, both passenger and carrier are
destroyed. Units traveling on armored units may
be attacked exclusive of the armored unit in
which case the passenger unit has a DF of 1.
The armored unit is unaffected if only its
passengers are attacked.

\item [J -]
Armored units may fire when loading, unloading
or carrying passengers. All other transport rules,
however, apply.

\item [K -]
Russian cavalry units may NOT be transported
by ``C'' units or armored units.
\end{enumerate}

\subsection{Road Movement}

Units traveling along roads do so at the road
movement rate regardless of the other terrain in the
road hexes. All units may travel over all roads of the
board regardless of accompanying terrain in which
the might otherwise be prohibited. Roads do not
alter the defense effects of surrounding terrain.

\begin{enumerate}
  \item [A -]
All units move along roads at a cost of 1/2
movement factor per road hex. Entering a road
hex through a non-road hex side is done at the
MF cost of the other terrain in the road hex
being entered.

\item [B -]
Units may freely combine road and non-road
movement in the same turn.

\item [C -]
Units may not stack while moving along road at
the road movement rate.

\item [D -]
Units may not move through or onto other units on
a road when moving at the road movement rate.

\item [E -]
To move through or onto a friendly unit on a road
costs the full non-road movement cost of the other
terrain in that hex, the moving unit incurs the non-
road MF cost of the other terrain in that hex as
well. In effect, you are ``passing'' the unit sitting on
the road b swinging off the road and maneuvering
around it. Terrain hexes or hex sides through
which a unit would be prohibited to travel when off
the road cannot be traveled upon when performing
this ``passing'' maneuver. For example a vehicular
unit could NOT move through a unit on a
swamp/road hex. It could, however. move ONTO
such a unit and move off in the next turn.

\item [F -]
  Units may stack with other units and ove along
roads at the NON-road movement rate (and a unit
in such a stack could split off and move ahead by
itself at the road movement rate).

\item [G -]
Remember: a vehicle with passengers is
considered as one unit and may therefore travel
together at the road movement rate.
\end{enumerate}

\subsection{Stacking (more than one unit per hex)}

\begin{enumerate}
  \item [A -]
The Russians may stack two units per hex.

\item [B -]
The Germans may stack three units per hex.

\item [C -]
When a unit is being carried by another unit, the
passenger and the carrier is considered as one
unit for stacking purposes.

\item [D -]
Stacking limitations do not apply during
movement. They only apply before and after
movement (except as qualified by the road
movement rules).

\item [E -]
Minefield counters and Fortification counters do
not count towards stacking limits.

\item [F -]
Block counters and Wreck counters ARE counted
towards stacking limits.

\item [G -]
Vehicle units may not stack with other vehicle
units on swamp-road hexes.
\end{enumerate}



\section{Combat}

\subsection{How To Have Combat}

\begin{enumerate}
  \item [A -]
Basically, to have combat, the attacking unit
compares its Attack Factor (AF) to the defending
unit's Defense Factor (DF). The comparison is
stated as a ratio: AF to DF; then rounded off in
the defender's favor to conform to the ratios
given on the Combat Results Table (CRT).
Example: 11 to 3 rounds off to 3 to 1. Roll the
die and take the action indicated by the CRT.

\item [B -]
Attack takes place before the movement portion
of a player's turn. Only the player whose turn it
is may attack, the other player is considered the
``defender.''

\item [C -]
Only enemy units within the Range Factor (RF)
of the attacking unit my be fired upon by that
unit.

\item [D -]
A player may make as many or as few attacks
per turn as he desires (within the restrictions of
the rules of combat). A player is never forced to
attack. Attacking is an act of volition.

\item [E -]
Every firing unit firing on the same defending
unit must combine their Attack Factors into one
large Attack Factor before computing odds.
Units may fire only once per turn.

\item [F -]
Units which fire (attack) in the combat portion of
the turn may NOT move in the movement
portion of the same turn.

\item [G -]
Different Attacking units may fire at the same
target unit. Each firing unit is announced to be
firing at a common target, and the combined
attack is resolved all at once.

\item [H -]
Units may not split their attack factor (i.e. a given
attacking unit could not apply part of its factor to
one attack and part to another). Attack and
defense factors are not ``transferable'' from one
unit to another. Each unit is treated as an
indivisible set of factors.

\item [I -]
When there is more than one defending unit in a
hexagon, attacks against that stack of units may
be prosecuted in one of the three mutually
exclusive ways:

\begin{enumerate}
  \item SELECTIVE ATTACK: Only one unit (the
weakest) in the defending stack is attacked; the
others are ignored and may not be attacked; the
others are ignored and may not be attacked in the
normal attack phase of that run. The ``weakest'' unit
is defined as being that unit in a defending stack
against which that particular attacking units(s) can
attack at the highest odds ratio. Selective attacks
may be made at odds as low as 1 to 4.

\item MULTIPLE ATTACK: More than one unit in the
defending stack is attacked. Each unit attacked is
treated as a separate battle with the die being
rolled once for each battle. In this type of attack the
WEAKEST unit is attacked first at at least 1 to 1
odds: finally the strongest unit (defense factor) may
be attacked (at at least 1 to 1). By definition, at
least TWO of the defending units in a stack must
be attacked to use this method. Of course, to be
able to make a Multiple Attack against a defending
stack, the attacker must use at least one different
attacking unit against each defender. Remember a
unit may never use its attack factor more than once
per turn.

\item COMBINATION ATTACK: The defending units are
treated as one combined defense factor and the
attacker totals his units into one combined attack
factor. The die is rolled once for the entire stack
and the result applies to all the defending units in
that stack. Combination attacks may be made by
as few as one attacking unit and at odds as low as
1 to 4. See the Weapons-to-Target Relationships
rule.
\end{enumerate}

\end{enumerate}

\subsection{Overrun Attack}

\begin{enumerate}
  \item [A -]
Armored vehicle units may overrun enemy units in
clear terrain.

\item [B -]
To overrun a unit or stack of units, move the
attacking armored vehicle unit(s) straight through
the enemy-occupied hex, exiting into the hex
DIRECTLY opposite the hex of entry. Overrunning
units must stop in the ``exit-hex'' and may move no
further that turn. If the exit hex is occupied by
enemy units, the overrun may not be made.
Overrunning units may not travel at the road-
movement rate during that turn. Overrunning units
must have sufficient movement factors available to
reach the exit hex. The exit hex does not have to
be a clear terrain hex, nor does the entry hex: only
the target hex must be clear terrain.

\item [C -]
As you move over the enemy unit or stack of units,
execute your attack. This is the only case in which
an attack may be made during the movement
phase of a turn.

\item [D -]
Overrunning units attack with an increased
combat effect. Figure the odds ratio of the attack
using the basic AF to DF system then increase
the odds by one in favor of the attacker ( e.g. a
ratio of 3 to 1 increases to 4 to 1). Also subtract
2 from the die roll results (e.g. a die roll of 3 is
treated as if it were a roll of 1). A defending
stack is treated as one combined defense factor
when being overrun. More than one armored
unit may overrun an enemy stack and the
overrunning units do not have to enter and exit
through the same hexes. They must, however,
combine their attack into one large attack factor.
In other words, a defending unit or stack may
not suffer more than one overrun attack per turn.

\item [E -]
In determining overrun odds use only the factors
printed on the units (for attacker and defender).
Do not halve or double the attack factors as
shown on the Weapons Effectiveness Chart.
Use only the overrun ``bonus'' as outlined in rule
``D''.

\item [F -]
You may not fire overrunning units during the
usual combat portion of the turn in which the
overrun is made.

\item [G -]
  Units on Block, Wreck, Minefield or Fortification
counters may not be overrun.

\item [H -]
The German SPA units (Maultier, Wespe and
Hummel) may NOT make overrun attacks.
Halftracks may not overrun Armored Vehicles
(including enemy halftracks).

\item [I -]
Units may only be overrun when they are in
clear terrain or clear terrain-road hexes.

\end{enumerate}

\subsection{Close Assault Tactics}

All types of Russian and German infantry and
engineer units as well as Russian cavalry units have
the option of using Close Assault Tactics instead of
making a normal attack. Close Assault takes place
AFTER movement.

\begin{enumerate}
  \item [A -]
Close-assaulting units must be directly adjacent
to the defending unit or stack of units (i.e. in one
of the six surrounding hexes).

\item [B -]
CAT attacks take place after all movement,
normal attacking and overrun attacks are
finished.

\item [C -]
Units utilizing CAT may NOT make normal attacks
in the same turn. They may, however, move in the
same turn. (NOTE: Overrun and CAT attacks are
the only exceptions to the general rule which
forbids movement and combat by the same unit in
the same turn.)

\item [D -]
Close Assaulting units have their effectiveness
increased by subtracting 2 from their die-roll result:

e.g. a die-roll of ``2'' becomes a die-roll of ``0''. The
defending stack must be treated as one combined
defense factor, and may suffer only one Close
Assault per turn.

\item [E -]
If infantry and engineer units are stacked together
when Close Assaulting the same defender, the
effectiveness of that Close Assault is further
increased by raising the odds in their favor to the
next highest ratio (as in the Overrun rule). At least
one engineer unit must be stacked with at least
one infantry or cavalry unit, on at least one of the
hexes of Assaulting units.

\item [F -]
Units capable of using Close Assault do not HAVE
to use it to attack adjacent enemy units (they may
attack them normally in the normal attack phase if
the player so desires).

\item [G -]
Close Assault is the only way in which ``I'' units
may attack Armored Vehicle units.

\item [H -]
Any type of defending unit may be attacked using
Close Assault Tactics, CAT may be used in any
type of terrain.

\item [I -]
Russian cavalry may not move more than one hex
in a turn in which it is to be used for a Close
Assault. No unit may use the road movement rate
and make a Close Assault in the same turn.

\item [J -]
Halftracks may not use CAT. (See optional rules for
halftrack assisted CAAT.)
\end{enumerate}

\subsection{Weapon-to-Target Relationships}

The class-key letter symbols determine what type of
weapons a unit is armed with.

``I'' = Infantry weapons (rifles \& machineguns) Note: Halftracks have ``I'' type weapons.

``A'' = Armor Piercing Weapons (high velocity tank \& antitank guns).

``H'' = High Explosive Shells (low velocity howitzers, etc.).

``M'' = Mortars (similar to ``H'').

\subsection{WEC Weapons Effectiveness Chart}

The effectiveness of these weapons changes in
relation to target type and range from target. This is
reflected in the WEC by doubling and halving a
unit's attack factor accordingly. Note: Half range is
always rounded off to the nearest whole hex; the
attacking unit loses fractional RF's (for example: half
of 9 is 4).

\begin{enumerate}
\item [A -]
When making a normal Combination Attack
against a mixed stack of units, determine what
type of target (Armored or Non-armored)
predominates in the stack and then treat the
entire stack as if all the units in it were that type
of target. If the target stack is divided evenly
between Armored and Non-armored targets,
treat the whole stack as if it were that type of
target least favorable to the particular attacking
unit(s).

\item [B -]
Units in towns are considered as Armored
targets whether or not such units actually are
armored. Units stacked together in towns MUST
be attacked as one combined DF.

\item [C -]
Fortifications are considered as Armored
targets.

\item [D -]
WEC is not used in determining overrun odds.
\end{enumerate}

\section{Obstacles and Elevations}

The PanzerBlitz mapboard is a two-dimensional
representation of a three dimensional space. The
various terrain features, aside from affecting
movement and defense, also affect the ability of
attacking units to fire at given defending units. Since
the weapons used in PanzerBlitz are direct-fire
weapons, an attacking unit may not fire at any target
which it cannot ``see.'' The terrain symbols on the
map show the location of potential obstructions and
the obstacle-hex side symbols show in which
direction fire is obstructed. These ``heavy'' hex-side
symbols are color coded according to the type of
obstruction they represent.

There are three general types of obstacle/hex-sides
which cut off the line of sight and therefore prevent
fire:

\begin{enumerate}
  \item LOW OBSTACLES: Ground-level Green (woods) hex-
sides and ground-level Gray (town ) hex sides. 10 to
20 meters.
  \item MEDIUM OBSTACLES: Dark Brown (Slope) hex-sides.
60 to 70 meters.
  \item HIGH OBSTACLES: Orange (hilltop hex-sides. 100 to
140 meters.
\end{enumerate}

Whether or not a firing unit can ``see over'' these
Obstacles depends upon the elevation at which the
firing unit and its potential target are. Units themselves
are not considered as obstacles, and players may fire
over or through all units, enemy or friendly.

There are three elevations at which a unit may be:

\begin{enumerate}
  \item GROUND LEVEL: 0 meters.
  \item SLOPE LEVEL: 50-60 meters.
  \item HILLTOP LEVEL: 100 to 140 meters.
\end{enumerate}

The elevation at which a unit is, is of course
determined by the terrain it is on. The Target Elevation
Table shows, in a general way, what hex-side symbols
obstruct the line-of-fire in different cases. The table,
however, does not cover all the situations which might
arise.

\subsection{How to Determine the Line of Sight/ Line
of Fire (LOS/LOF)}

For all practical purposes the Line-of-Sight is
equivalent to the Line-of-Fire. In ``real life'' the Line-of-
Sight would be a perfectly straight line. In the game,
however, the Line-of-Sight is traced through the
hexagons in a manner similar to the way units are
moved. First, determine the range to the target. Then
plot the route through the hexagons which your Lineof-
Sight takes. The Line-of-Sight should trace through
the exact same number of hexes as the range to the
target. The Line-of-Sight, in other words, should be as
short and as straight as possible while still conforming
to the hexagonal grid. Whenever there is more than
one possible ``shortest and straightest'' route, choose
the one least favorable to the attacker: i.e. if one route
is clear and an alternate route is obstructed, the
obstructed route is considered to be the one through
which the Line-of-Sight must be traced. In other words,
the defender gets the benefit of the doubt. Using the
TET and the special notes, determine whether or not
any hex-side symbols encountered in tracing the Lineof-
Sight, actually obstruct the Line-of-Sight.


\subsection{Special Notes (To Be Used In
Conjunction With TET)}

\begin{enumerate}
  \item [A -]
When firing FROM a slope or hilltop TO a
ground-level target, the Line-of-Sight is
obstructed if the target unit is directly behind a
Gray or Green hex-side: i.e. ``directly behind''
means that the intervening Gray or Green hex-
side(s) forms one or more of the hexagon sides
of the target hex itself.

\item [B -]
When firing FROM ground-level TO a target on
a hilltop or a slope, the Line-of-Sight is
obstructed if the FIRING UNIT is directly behind
a Gray or a Green hex-side.

\item [C -]
When firing FROM a hilltop TO a ground-level
target, the Line-of-Sight is obstructed by
intervening Brown hex-side symbols if such a
symbol is closer to the TARGET UNIT than to
the firing unit or if the symbol is exactly mid-way
between the two. To determine the relative
position of the Brown symbol, count the number
of hexagon SIDES through which the Line-of-
Sight is traced (including the side of the firing
hex and the side of the target hex).

\item [D -]
When firing FROM ground-level TO a target on
a hilltop, the Line-of-Sight is obstructed by
intervening Brown hex-side symbols if such a
symbol is closer to the FIRING UNIT than to the
target, or exactly midway between the two.
(Note B is the converse of Note A and Note D is
the converse of Note C.)

\item [E -]
The ONLY case in which a unit may trace an
unobstructed Line-of-Sight through more than
ONE Orange (hilltop) hex-side symbol is when
both the target and the firing unit are on hilltops.
In all other situations, the LOS is obstructed if it
must be traced through more than one Orange
hex-side.

\item [F -]
No matter what the obstacle or the terrain, a unit
may ALWAYS fire at a target to which it is
directly adjacent (regardless of elevation).

\item [G -]
Notice that in some cases there are towns and
woods on top of hilltops. The Green and Gray
symbols in these cases obstruct ALL fire, no
matter what the elevation of the target and the
firing unit (except when directly adjacent to each
other as per note ``F'').

\item [H -]
Note that when both the target and the attacker
are on ground-level ALL hex-side symbols

(Gray, Green, Brown, and Orange) obstruct the
Line-of-Sight (except as per note ``F'').

\item [I -]
(MAPBOARD NOTES:) The interior hexes on the
large plateau-like hilltop on board No.2 are all
hilltop hexes even though they do not have
Orange hex-sides superimposed upon them. For
practical and esthetic purposes the board designer
felt it would have been redundant to so outline
those hexes. You will notice that in some cases,
slope hexes do not have Brown hex-side symbols.
This is because the Brown symbols actually
represent the way in which slopes curve and form
``corners'' to obstruct the LOS. Consequently,
whenever a slope is relatively straight, it does not
obstruct the LOS along it. In some instances, two
or more slope hexes meet without a hilltop being
formed between them. This represents a ridge or
``razorback'' hillock. The Brown hex-side symbols
represent the ``spine'' of these ridges.
\end{enumerate}

\subsection{Hill and Slope Defense Exceptions}

As indicated on the TEC (Terrain Effects Chart) a unit
attacking an enemy unit defending on a slope or
hilltop, attacks at half-attack-factor. There are some
exceptional cases, however, in which the attacker is
NOT halved. They are:

When the defending unit is on a hilltop, an attacking
unit is NOT halved if it is also on a hilltop (not
necessarily the same hilltop). A defender on a hilltop
could conceivably be attacked by units not on hilltops
and units on hilltops as part of the same attack; in
which case the attackers not on hilltops would STILL
be halved.

\begin{enumerate}
  \item [A -]
When a defender is on a slope, the attacking unit
is NOT halved if the attacker is directly adjacent to
the defender (regardless of elevation). If, however,
there is a Brown hex-side symbol between an
adjacent attacker and defender, the attacking
unit's factor IS halved. Units on slopes can
conceivably be attacked by attacking units which
are halved and which are not halved as part of the
same attack.

\item [B -]
In all situations other than those described in ``A''
and ``B'' above, an attacking unit (regardless of
elevation is halved when firing at units defending
on slopes or hilltops.
\end{enumerate}


\subsection{Gullies and Streambeds}

The gullies and streambeds shown on the board are
DEPRESSIONS (minus 5 to 7 meters). The terms
``streambeds'' and ``gullies'' are interchangeable.

\begin{enumerate}
\item [A -]
Units in gullies may not fire at (or BE fired at by)
units at ground level or in other gully-hexes
(unless they are directly adjacent to each other).

\item [B -]
Units in gullies may fire at (and BE fired at by)
units on slopes and hilltops. In these cases treat
the unit in the gully as if it were at ground-level
for TET purposes.

\item [C -]
``Fords'' represent exposed (not depressed)
areas of a streambed. They are equivalent to
clear terrain for all purposes. The hexagons on
which a road crosses a gully are also
considered as clear terrain (even if for some
reason the road becomes unusable).

\item [D -]
Hexagons containing the end or beginning of a
gully are treated as full fledged gullies.
\end{enumerate}

\subsection{Spotting}

When a defender is in a Woods or a Town hex, he
may not be fired upon by units which are not directly
adjacent to him unless he has been ``spotted.''

\begin{enumerate}
  \item [A -]
To ``spot'' a unit in a town or woods hex, the
attacker must have a friendly unit directly
adjacent to the defender.

\item [B -]
The unit doing the spotting does not necessarily
have to be involved in the attack upon the
spotted unit.

\item [C -]
Dispersed units may not be used for spotting.

\item [D -]
The same unit may spot any number of adjacent
enemy units.
\end{enumerate}

\subsection{Wreckage}

Whenever an ARMORED vehicle unit (including
halftracks) is destroyed in combat, remove it from
the board and replace it with a Wreck counter.

\begin{enumerate}
  \item [A -]
Wrecks may not be moved or removed.

\item [B -]
Wrecks count as a unit for stacking purposes.

\item [C -]
A Wreck on a road NEGATES THE ROAD, and
that hex is treated as a non-road hex. Vehicle
units may not enter swamp/road hexes

containing Wrecks, nor may they cross Green-hex
sides from a road hex with a wreck in it.

\item [D -]
The presence of Wrecks has no effect on combat.

\item [E -]
It is possible to have as many as three German or
two Russian Wrecks in a hex, or a combination of
German and Russian Wrecks not exceeding a
total of THREE.

\item [F -]
Units may enter hexes containing enemy and/or
friendly Wrecks as long as they do not exceed
stacking limits.
\end{enumerate}

\subsection{Mines}

\begin{enumerate}
  \item [A -]
The player with the minefield pieces positions
them anywhere on the board he desires unless
otherwise directed by the Situation Card. One
mine per hex.

\item [B -]
Once positioned, mines may not be moved.

\item [C -]
Minefields have no friends � they affect both sides.

\item [D -]
As soon as a unit moves onto a minefield, it must
stop.

\item [E -]
The opposing player, during the combat portion of
his turn rolls the die for the ``attacking minefield.
The minefield attacks ALL units at 2 to 1 odds, no
matter what the terrain. Surviving units may move
off in there next turn.

\item [F -]
A minefield is never ``used up.'' It remains active
until removed by an Engineer unit.

\item [G -]
Engineers remove mines by moving adjacent to
them and on the turn after moving adjacent rolling
the die. A roll of ``1'' removes the mines. If
unsuccessful, the Engineers may remain adjacent
roll each turn until they get a ``1''.

\item [H -]
Units ``dispersed'' by minefields may not move off
them, and suffer minefield attack in their next turn
again.

\item [I -]
Minefields do not count against staking limits.

\item [J -]
Minefield ``attacks'' take place BEFORE normal
attacks.
\end{enumerate}


\subsection{Positional Defenses}

\section{Blockage}

The BLOCK counters represent tank traps, road
blocks, barbed wire, felled trees and anything else
that might impede movement.

\begin{enumerate}
  \item [A -]
Blocks may be placed anywhere on the board,
no more than one Block per hex. Once placed
they may not be moved or removed.

\item [B -]
A unit may only enter a hex with a Block in it if it
begins its turn directly adjacent to the Block.
Upon entering the Block-hex, the unit must stop
and go no further that turn. In their NEXT turn
they may move off the Block at the normal
movement rate.

\item [C -]
Blocks do not obstruct the Line-of-Sight and
have no effect upon combat.

\item [D -]
When a Fortification is destroyed, replace it with
the Block counter.

\item [E -]
A Block counter on a road negates the road in
that hex. VEHICULAR UNITS may not enter
Block-road hexes in Swamps nor traverse
directly adjacent Green hex-sides when moving
off a block road hex.
\end{enumerate}

\subsection{Fortifications}

(Bunkers, Redoubts, prepared trenches, etc.)

\begin{enumerate}
  \item [A -]
Fortifications are place wherever on the board
the player desires (except swamp hexes) or as
directed by the Situation Card. Once placed they
may not be moved.

\item [B -]
Fortifications do not affect movement or
stacking.

\item [C -]
Units in fortifications defend using the defense
factor of the fortifications only. Their own DF's
do not count. Fortifications are treated as an
ARMORED target. Terrain and Weapons
Effectiveness are taken into account when a
fortification is defending. Any type of unit(s) may
occupy a fortification within normal stacking
limitations.

\item [D -]
Units may fire from fortifications using their
normal attack factor (AF).

\item [E -]
If a fortification is destroyed, any units in it are
also destroyed.

\item [F -]
The fortification counter itself, has no attack factor
and it my only defend.

\item [G -]
If abandoned or unoccupied, fortifications may be
``captured'' and used by the opposing player. To
capture a fortification, simply move a unit into the
unoccupied fortification counter.

\item [H -] Unwanted fortifications may only be destroyed by
attacking them with one's own fire weapons.

\item [I -]
The fortification unit is not an obstacle to fire,
whether units are in it or not.

\item [J -]
Units are indicated as being IN the fortification by
placing them UNDER the fortification counter.
Friendly units ON TOP of fortifications (i.e.
``outside'' the fortification) and the fortification
counter on which they are sitting, may be attacked
using any one of the three Normal attack
techniques as well as Close Assault Tactics.

\item [K -]
Enemy units may be actually ON the fortification
counter while friendly units are still in it. In this
case you may fire on that hex but you must roll the
die twice for this attack, once for the attack on the
enemy units and then for the effect on your own
fortified units. In this case it is possible to destroy
the enemy unit and not your own, or both. Or you
may simply have a combination of dispersals.
Enemy units may not enter a fortification hex if
there are friendly units ON TOP of the fortification.

\item [L -]
When enemy units are sitting on top of a
fortification, friendly units inside the fortification
may not leave. The friendly units may still attack
any enemy units within range, including the units
sitting on the fortification: when attacking the units
sitting on the fortification the units inside attack as
if they were adjacent to the enemy. Similarly, the
enemy units may attack the fortification as if they
were adjacent to it. The units inside, in this case,
may NOT use Close Assault Tactics; the enemy
units on top may use CAT.

\item [M -]
  Fortifications may suffer ``dispersal'' in which case
the units occupying them are dispersed also.

\item [N -]
No more than one fortification may be placed in a
given hex.

\item [O -]
Armored vehicles may NOT make Overrun Attacks
against fortifications.

\item [P -]
  Destroyed fortifications are replaced with block
counters.

\item [Q -]
Fortifications may NOT be placed on top of Mines.
\end{enumerate}


\section{Game Procedure}

\subsection{Sequence of Play}

The game is played in turns, each player moving
and having combat sequentially. Two ``Player-Turns''
equals one complete ``Game-Turn''.

\subsection{German Player Turn}

\begin{enumerate}
  \item [Step 1] German player resolves any Minefield
attacks against Russian units.

\item [Step 2] German player announces which of his units
are attacking which Russian units, and what
attack techniques are being used.

\item [Step 3] German player resolves all Normal combat,
rolls the die once for each attack. German
player flips face-down all firing units, as they
are fired, to signify that they may not move.

\item [Step 4] German player moves as many face up
VEHICULAR units as he desires, executing
any Overrun attacks as he does so.

\item [Step 5] German player moves any face-up NONVEHICULAR
units and makes Close Assaults
after doing so.

\item [Step 6] German player turns ALL his units FACEUP,
including those dispersed by Russian
attacks in the previous turn.
\end{enumerate}


\subsection{Russian Player Turn}

\begin{enumerate}
  \item [Step 7] The Russian player repeats Steps 1 through
6 using his own units.

\item [Step 8] Indicate the passage of one complete
Game-Turn on the Turn Record.
\end{enumerate}

Players repeat steps 1 through 8 for as many turns
as the Situation Card indicates or until one player
concedes.


\section{Optional Rules}

Players may employ as many or as few of the
Optional Rules as desired.

\subsection{Indirect Fire}

German SPA units (Maultier, Wespe and Hummel)
and all German and Russian mortar units 9M) may
use Indirect Fire: i.e. they may fire at units which they
cannot ``see''; firing over all obstacles to the limit of
their ranges.

\begin{enumerate}
  \item [A -]
Indirect Fire may only be used in conjunction with
CP units (200's).

\item [B -]
The CP unit must be able to see the target: i.e.,
trace a clear Line-of-Sight to the target.

\item [C -]
Targets in towns or woods must still be ``spotted''
(but not necessarily by the CP unit itself). The
spotting unit would radio or flare-signal the CP,
which in turn would radio the target information to
the unit firing indirectly.

\item [D -]
SPA units may only use Indirect Fire against
targets which are at more than half the SPA unit's
range.
\end{enumerate}

\subsection{Real Space Line of Sight Determination}

Use a straight-edge (ruler, stiff cardboard, etc.) to
determine the Line of Sight. Determine range in the
normal way but to calculate Line of Sight, place a
straight edge between the center of the firing hex and
the center of the target hex. Only those hex-side
symbols intersected by the straight-edge need be
taken into account. If the straight edge bisects a hex-
side symbol through its LENGTH, take that symbol
into account UNLESS it is a BROWN symbol which
connects with an Orange symbol. The defender gets
the benefit of the doubt if the straight-edge cuts
exactly through the ``corner'' of a hex where a symbol-
side and a non-symbol side meet. In any other
ambiguous cases, use ``common-sense'' to decide,
keeping in mind the fact that the board is representing
a three-dimensional space.

\subsection{Panzerblitz Assault}

Infantry units riding on Armored vehicles may ``jump
off'' in the hex immediately in front of a unit about to be
Overrun by those Armored vehicles. The infantry may
then Close Assault the units which were just Overrun
by the vehicles.



\subsection{Ammunition Rule}

Anytime he desires, a player may fire one or more of
his ``H'' or ``M'' class units ``intensively.'' Intensive fire
allows such units to TRIPLE their normal attack
factor.

\begin{enumerate}
  \item [A -]
A given unit may fire intensively only once per
game, immediately after which it is removed
from play (its ammunition has been expended
and its gun tubes burned out).

\item [B -]
Units removed from play under this rule are
counted as units lost.

\item [C -]
Players may exercise this intensive fire option as
many times as they wish during a game, limited
only by the quantity of ``H'' and ``M'' units
available.

\item [D -]
``H'' CLASS ARMORED VEHICLE UNITS MAY
NOT USE INTENSIVE FIRE AS PART OF AN
OVERRUN ATTACK.

\item [E -]
MAULTIER unit may not use intensive fire.
\end{enumerate}

\subsection{Experimental Rules}

The following experimental rules are just that:
experiments! The game factors and mapboard are
not necessarily designed to accommodate them.

\subsection{Hidden Deployment}

Utilizing one of the low unit-count Situations or one
of your own devising, allow one player to secretly
deploy his units on the board, marking their positions
on separate pieces of paper using the Map Location
System, and remove the units from the board. The
other player is not shown where a given enemy unit
is until it fires its weapons or until the second player
has a clear Line of Sight to the unit's position. Once
hidden units are moved, fired or spotted, they must
be placed on the board and left in view. Minefields
may be hidden anywhere; combat units may only be
hidden in woods, swamps, towns, or behind hills.


\subsection{Experimental Indirect Fire}

Allow any German unit(s) to perform the fire-direction
function of a CP unit. Allow Russian
Guards infantry this capability.



\subsection{Impulse Movement \& Return Fire}

This rule is a ``meshing'' of the two player turns which
breaks down the simulated event into smaller bits of
time called ``impulses.'' Players use the following
sequence for a complete turn:

\begin{enumerate}
  \item [gA] German attacks.  moves units, executes Minefield
  \item [gB] Russian fires (do not flip firing units).
  \item [gC] German fires (does flip firing units).
  \item [gD] German moves non-firing units, executes Overruns.
  \item [gE] German makes Close Assaults, flips ALL units right-side-up.

  \item [rA] Russian moves units, executes Minefield attacks.
  \item [rB] German fires (do not flip firing units).
  \item [rC] Russian fires (does flip firing units).
  \item [rD] Russian moves non-firing units, executes Overruns.
  \item [rE] Russian makes Close Assaults, flips ALL units right-side-up.
\end{enumerate}


In this system, all units with a movement factor of
MORE than 1 may only move HALF their movement
factor in any movement impulse. Units with an ODD
movement factor use the greater ``half'' of their factor in
impulse ``D.'' Units with a movement factor of 1 may
move one hex EACH movement impulse (A\&D) but
they may not move at the road movement rate in
impulse ``A.''


\section{House rules}

The following house rules are gleaned from a wide variety of sources.
The philosophy for this rule set is to provide more simulation value
without compromising game play, and without grossly affecting the original
situations.

If all of these house rules are implemented during a game, any rule
benefitting one side over another should allow for modifying victory
conditions to reflect the effect of the rule change.

\subsection{Light artillery supported CAT (LASCAT)}


``75mm infantry guns (2 H 12) and company mortars (3 M 12) may add their factors
in to CAT infantry attacks if within range. Infantry guns require LOS; mortars
may fire indirectly.''
\href{http://boardgamegeek.com/article/15828184#15828184}{CAT support}

This is for German infantry CAT (optionally CAAT) only.

Russian LASCAT limited to Guards units.

This rule will have minor effect for an inexperienced player. For scenarios
where the balance favors the Russians, this rule may provide some balance
for an experienced German player.


\subsection{CAAT}

Light armor assisted CAT, one way to use halftracks in close assault.


\subsection{Balkas}


Balkas are a certain kind of gully in the Ukraine.

Experimental:

\begin{itemize}
\item Units exploiting balkas for concealment may not stack.
\item Units attacking out of a balka pay +1 DRM.
\item Units defending in balka suffer -1 DRM.
\end{itemize}

These simple modifications could go a long way towards balancing up scenarios
such as presented in Situation \#2, where the Russians can basically sneak down
the gullie for a certain win.

\section{FAQs and more}

This section is both Frequently Asked Questions and ``collected wisdom'' grouping
rules by unit or counter or in some other way than presented in the original rules.


\subsection{Halftracks}

Rules specific to halftracks are scattered through many sections of the
original rules. Here is a handy list of halftrack-specific rules.

\begin{itemize}
\item May overrun infantry and non-armored transport.
\item Replaced by wreck counter when destroyed.
\item May not be used for CAT in original rules.
\item May be used for CAAT under optional rules.
\end{itemize}

Halftracks were widely used during World War II, and it makes sense
to find a way to use them in PanzerBlitz as something more than
panzer fodder.

\subsection{Road march order}


Many scenarios call for units to enter from a side of a map board, or
specifically, to enter using a road. In any case where units enter via a road,
``road march order'' is assumed in force.  Road march order simulates the
movement of a column of vehicles traveling along a road.


Recall that roads limit stacking to one unit, where transporting and
transported units are considered as a single unit. Passing along roads is
subject to movement rules governing surrounding terrain.

[Figure here]


\end{document}
