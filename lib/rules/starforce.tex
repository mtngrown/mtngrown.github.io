\documentclass[10pt]{article}

\def\bookauthor{Dave Doolin}
\def\booklongtitle{Olympica}
\def\wwkeywords{olympica, wargame, lynn-willis}
\def\wwsubject{UN on Mars}
\def\wwslug{olympica}
\def\baseurl{http://dool.in}

%%% Common include file for game rules

\setlength{\oddsidemargin}{0.in}
\setlength{\textwidth}{6.5in}
\setlength{\topmargin}{0.in}
\setlength{\textheight}{8.5in}

\usepackage{graphicx}
\usepackage{url}
\usepackage{amsmath}
\usepackage{moreverb}
\usepackage{draftcopy}
\usepackage{multicol}
\usepackage{boxedminipage}
\usepackage{titlesec}
\usepackage{enumitem}
\usepackage{listings}
\usepackage{textcomp}
\usepackage[table]{xcolor}
%\usepackage{colortbl}

\RequirePackage{fancyhdr}
\RequirePackage[colorlinks]{hyperref}

\RequirePackage{fancyvrb}
\DefineVerbatimEnvironment{code}{verbatim}{fontsize=\small}

\RequirePackage{color}
%%% Find colors: http://en.wikipedia.org/wiki/List_of_colors
\definecolor{seagreen}{rgb}{0.18, 0.55, 0.34}
\definecolor{orangy}{cmyk}{0.0, 0.58, 0.100, 0.10}
\definecolor{saffron}{rgb}{0.96, 0.77, 0.19}
\definecolor{carmine}{rgb}{0.59, 0.0, 0.09}
\definecolor{chocolate}{rgb}{0.48, 0.25, 0.0}
\definecolor{darkraspberry}{rgb}{0.53, 0.15, 0.34}
\definecolor{electricindigo}{rgb}{0.44, 0.0, 1.00}
\definecolor{oldmauve}{rgb}{0.40, 0.19, 0.28}
\definecolor{blue-violet}{rgb}{0.54, 0.17, 0.89}
\definecolor{blue(pigment)}{rgb}{0.2, 0.2, 0.6}

\hypersetup{%
  pdfauthor={\bookauthor},
  pdftitle={\booklongtitle},
  pdfkeywords={\wwkeywords},
  pdfsubject={\wwsubject},
  urlcolor=cyan,
  pdfdisplaydoctitle=true,
  pdfcreator={Website In A Weekend},
  pdfproducer={\LaTeX},
  baseurl={\baseurl/\wwslug},
  breaklinks=false,
}


\title{StarForce `Alpha Centauri': Interstellar Conflict in the 25th Century\\
\vspace{2 mm} {\Large Commentary and clarifications on Rules of Play}}
\date{\today}
\author{mtngrown}

\def\sf{{\em StarForce}}

\begin{document}

\maketitle

\tableofcontents

\begin{multicols}{2}

\section{Introduction}

\sf, a ``future wargaame'' designed by Redmond Simonsen, is an interesting
and still well-regarded game.

\section{Game Components}
\section{Preparation For Play}
\section{Placeholder}
\section{Sequence Of Play [Basic]}
\section{Planets}
\section{Economics}
\section{Political Events}
\subsection{National Interest Level}
\subsection{Colonization}
%\section{}
%\section{}
%\section{}

\section{FAQs and more}

\section{Summary}

\end{multicols}

\section*{Player Aid}

\begin{itemize}
  \item Full size True Distance Table
\end{itemize}

\section*{Kit}

\section{Errata}

The following from \href{http://www.russgifford.net/db_pages/game_starforce.htm}{Russ Gifford}:

\begin{quote}
Known Errata:
Source: Moves 21 (page 18)

Date: As of June 1, 1975

StarForce Errata

In early editions of StarForce, certain values in this Scenario were in error.
These are corrected below.

[38.0] THE RESCUE MISSION
A Solitaire Game

GENERAL RULE:

Players have twelve Star Forces available to lift off 60 Population Points
(each Population Point represents a third of a million humans). A combined set
of two Decimal Randomizers is used to simulate the uncertain time of nova. The
actual star in question is determined randomly at the start of the game.
Players win the situation by getting all the population safely off the planet.

[38.1] DETERMINING THE LOCATION OF THE ENDANGERED SYSTEM

Pick a chit from the Stellar Randomizer and read the top two-digit number. Read
that number as one of the hexes in the "2000" column of hexagons (the same
column that Sol and 70 Ophiuchi are in).

If the bottom number of the chit is positive trace a clockwise orbit around Sol
maintaining a constant hex distance from it (this will describe a large
hexagonal circle just like the rings of Zulu Limits printed on the map). If the
bottom number is negative. trace the orbit counter-clockwise. The first
tertiary star system that the orbit traces through (in a two-dimensional sense)
is the endangered system. If there is no tertiary system in that orbit. pick
another chit.

[38.2] INITIAL STARFORCE DEPLOYMENT

Four Star Forces at 202010 (Sol), two Star Forces at 2336/+ 17 (Sigma
Draconis), four Star Forces at the endangered star system. and two Star Forces
at the (undestroyed) tertiary system nearest to the endangered star (in true,
distance). If two stars are equally near. use the one which is also nearest to
Sol. All Star Forces are empty. All systems have StarGates except the
endangered star and those destroyed in the First Incursion (see 31.62).

[38.5] VICTORY LEVELS

Victory is measured in terms of how many Population Points are saved (each
equaling one Victory Point;. A perfect score of 60 Victory Points is a Decisive
Victory over the situation; a score of 50 to 59 is a Substantive Victory; 40 to
49 is a Marginal Victory. Less than 40 Points is a defeat. If a Star Force is
lost in the rescue attempt (either by Overshift results or being incinerated)
subtract three points from the score. Don't forget to count as lost any
Population Points on destroyed Star Forces at the time of destruction.
\end{quote}

\end{document}
